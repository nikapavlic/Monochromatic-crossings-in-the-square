\documentclass[a4paper,12pt]{article}
\usepackage[slovene]{babel}
\usepackage[utf8]{inputenc}
\usepackage[T1]{fontenc}
\usepackage{lmodern}
\usepackage{amsmath}
\usepackage{amssymb}
\usepackage[shortlabels]{enumitem}
\usepackage{graphicx}
\pagestyle{plain}

\begin{document}
\author{Nino Cajnkar in Nika Pavlič}
\date{December 2022}
\title{Monochromatic crossings in the square}
\maketitle

\section{Problem}
Naj bo $P$ množica $n$ enakomerno naključno izbranih točk v kvadratu. Vsak par točk povežemo z daljico.
Tako dobimo poln graf z $n$ vozlišči.
Naj bo $\lbrack a,b\rbrack$ interval kotov. Če je kot med daljico v grafu in x-osjo na intervalu
$\lbrack a,b\rbrack$, daljico obarvamo rdeče, sicer modro.

Zanima nas kje so presečišča med daljicami monokromatična (tj. obe daljici, ki se sekata sta enake barve)
in kje so bikromatična (tj. daljici, ki se sekata sta različne barve) ter kako je to odvisno od $a$ in $b$.

Cilj je najti tudi interval kotov $\lbrack a,b\rbrack$, ki da najmanjše število monokromatičnih presečišč.

Zanima nas še, kako se rezultati spremenijo, če so točke v enotskem krogu.

\section{Načrt dela}

Naloge se bova lotila s simulacijami in sicer bova opazovala kako se spreminjajo pozicije monokromatičnih in bikromatičnih
presečišč daljic s tem, ko se spreminja interval $\lbrack a,b\rbrack$ in ko se povečuje število točk $n$. Prav tako bova v simulacijah spreminjala interval $\lbrack a,b\rbrack$ 
in na podlagi razultatov poskušala določiti najmanjši tak interval, ki da najmanjše število monokromatičnih presečišč.
Najprej bova to naredila s točkami generiranimi v kvadratu $\lbrack 0,1\rbrack \times \lbrack 0,1\rbrack$, potem pa še s točkami znotraj enotske krožnice.

Generirane točke bova povezala v poln graf in povezave obarvala glede na prej opisan način. Za vsaki dve daljici bova preverila ali se
ti sekata. V kolikor se sekata bova poiskala presečišče in ugotovila ali je monokromatično ali bikromatično.
Potem pa bova opazovala v kje se monokromatična oz. bikromatična presečišča nahajajo in koliko jih je.

Pri reševanju problema bova uporabljala program R in Python.
\end{document}

